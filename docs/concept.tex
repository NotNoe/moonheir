\documentclass{article}
\usepackage[spanish]{babel}

\title{Documento de concepto} % Sets article title
\author{Noelia Barranco Godoy
\and Erik Zubimendi Solaguren\and
Elena Novillo Luceño}
\date{\today} % Sets date for date compiled


\begin{document} % All begin commands must be paired with an end command somewhere
    \maketitle % creates title using information in preamble (title, author, date)
    
    \section{INFORMACIÓN BÁSICA}
    
    \begin{tabular}{||c|c||}
        \hline
        \textbf{Título} & Moonheir \\
        \textbf{Género} & RPG narrativo \\
        \textbf{Plataforma} & Navegador web\\
        \textbf{Público objetivo} & Jugadores casuales\\
        \hline
    \end{tabular}

    \section{DESCRIPCIÓN}
    La historia comienza cuando nuestra protagonista, Seleni, una librera albina,
    decide completar el libro que le dejó su padre justo cuando la abandonó en el monte Upo.
    El objetivo es recuperar las páginas del libro, para ello tendrá que viajar a distintas
    zonas de España y superar algunos contratiempos.
    \\ \\
    Este juego es un RPG narrativo ambientado en la cultura local, lo
    que lo convierte una mezcla única. Es la combinación de una historia interesante, educativa
    y unas mecánicas de combate de RPG.

    \section{AMBIENTACIÓN}
    Nuestro juego está basado en la leyenda gitana \textit{"Hijos de la Luna"}
    , esto es importante para Seleni porque quiere entender sus orígenes.
    La estética del juego gira en torno a la noche y la Luna, por lo que a lo
    largo de la historia, se irá haciendo de noche.

    \section{MECÁNICAS PRINCIPALES}
    \begin{itemize}
        \item Combate RPG.
        \item Tipos de equipo estilo piedra-papel-tijera.
        \item Moverse entre distintas zonas para buscar las páginas.
        \item Desbloquear distintos tipos de equipamiento.
        \item Cambiar la equipación entre los tres tipos.
        \item Obtener consumibles.
        \item Ver las páginas del libro desbloqueadas.
        \item Aumentan tus estadísticas al completar una zona.
    \end{itemize}

    \section{REFERENCIAS}
    \begin{itemize}
        \item Hijo de la Luna. Mecano, 1986.
        \item Romancero Gitano. Federico García Lorca, 1928.
        \item Moonlighter. Digital Sun, 2018.
        \item Omori, OMOCAT, 2020.
        \item Pokemon Púrpura, Game Freak, 2022. 
    \end{itemize}

\end{document}
