\documentclass{article}
\usepackage[spanish]{babel}

\title{Documento de concepto} % Sets article title
\author{Noelia Barranco Godoy \\ 20532538R
\and Erik Zubimendi Solaguren \and Elena Novillo Luceño} %TODO: Poner nombres completos y DNI's
\date{\today} % Sets date for date compiled


\begin{document} % All begin commands must be paired with an end command somewhere
    \maketitle % creates title using information in preamble (title, author, date)
    
    \section{INFORMACIÓN BÁSICA}
    
    \begin{tabular}{||c|c||}
        \hline
        \textbf{Título} & \\
        \textbf{Género} & \\
        \textbf{Plataforma} & Navegador web\\
        \textbf{Público objetivo} & \\
        \hline
    \end{tabular}

    \section{DESCRIPCIÓN}
    This is the most important part. Most readers won’t really read past this point. You have at most two paragraphs to sell the idea. Focus on:
    1. What is the game about?
    2. Why make this game instead of playing something else / something better?

    \section{AMBIENTACIÓN}
    This section is about the narrative, thematic and aesthetics aspects of the game. 
    Again, two paragraphs at most. If there is no narrative, or the narrative aspects are not important, do not even mention them. 
    The whole section is OPTIONAL.

    \section{MECÁNICAS PRINCIPALES}
    This section is a LIST of the main game mechanics. Use at most two or three sentences for each entry.

    \section{REFERENCIAS}
    Games, visual media, books... 
    ANYTHING that helps understand the game by reference,
    or that inspired the game itself. Too many references are pointless,
    use like two or three, tops.

\end{document}
